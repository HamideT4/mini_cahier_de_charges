\documentclass[a4paper,11pt]{article}
\usepackage[T1]{fontenc}
\usepackage[utf8]{inputenc}
\usepackage{lmodern}
\usepackage[frenchb]{babel}
\usepackage{xcolor}
\usepackage{graphicx}
%\usepackage{float}
\usepackage{capt-of}


\title {\textcolor{cyan}{\textbf{CAHIER DES CHARGES DU PROJET DE E-BOUTIQUE}}}
\author{HAMIDE MAHAMAT}

%\addto\captionsfrench{
 % \renewcommand{\partname}{Chapitre}
 % \renewcommand{\thepart}{\Roman{part}}
%}

\begin{document}

\maketitle
\thispagestyle{empty}

\newpage
\pagenumbering{roman}
\tableofcontents
\newpage
\listoffigures
\newpage
\begin{abstract}
Vue la diversité des ressources de la gestion des boutiques, l’évolution des méthodes de traitement des informations est devenue, de plus en plus, un besoin fondamental et indispensable.
\newline La stratégie de mettre en place un système d’informations, aura comme but d’assurer l’accélération et la précision lors du traitement des informations en garantissant un niveau de
sécurité et stabilité afin de faciliter les tâches de gestion et les rendre plus efficaces.
\newline Dans ce cadre, nous aspirons à automatiser les activités des boutiques en exprimant leurs différents besoins. À ce niveau, la gestion du personnel, des ventes, du stock, des dettes et des congés sont les besoins prioritaires ciblés, vue l’importance de quantités d’informations à traiter dans l’absence d’un vrai système de gestion.
\newline Le système d’information à réaliser pour assurer la gestion des boutiques, doit diminuer la quantité des charges du travail aux responsables du service personnel et faciliter l’échange des
informations entre les différents employés, cette stratégie doit aussi automatiser et accélérer la réalisation des tâches au sein de la boutique.
\newline Dans ce contexte, et pour la réalisation de notre projet, on a pris en charge de concevoir et réaliser une application permettant de faciliter la gestion des boutiques.
\newline
\newline Pour le faire, le travail doit être réalisé en deux parties ; une première partie théorique qui englobe l’étude, l’analyse et la conception du projet. Et une deuxième partie pratique qui vise à la réalisation concrète du projet. Ces deux parties seront préparées sur quatre points :
\begin{itemize}
  \item \textbf{Point 1}: Etude de faisabilité 
  \newline Il consiste, principalement, de présenter le contexte du projet, ses objectifs ainsi que le périmètre du projet.
  
  \item \textbf{Point 2}: Spécification des bésons
  \newline Dans cette partie, on se concentre sur la définition des différents types des besoins : fonctionnels et non fonctionnels (dites techniques), et l’extraction des différents futurs utilisateurs du système, pour arriver enfin à mettre une vue sur l’utilisation générale du système.
  \item \textbf{Point 3}: Conception
  \newline Dans ce chapitre, on va mettre en évidence une étude conceptuelle pour l’application tout en se basant sur le langage de modélisation unifié UML.
  \item \textbf{Point 4}: Mise ne oeuvre et réalisation
  \newline Ce dernière partie servira à présenter le travail réalisé, et les difficultés rencontrées.
 \end{itemize} 
 
 À la fin, on doit réaliser une conclusion générale qui résume le travail réalisé, et apprécie les
compétences acquises durant cette expérience.

\end{abstract}

\newpage
\pagenumbering{arabic}
\section{Etude de faisabilité}
\subsection*{Introduction}
\addcontentsline{toc}{subsection}{Introduction}
  Dans cette partie, nous allons présenter le contexte du projet, ses objectifs ainsi que son périmètre.
\subsection{Contexte du projet}
La réalisation de l'application web e-boutique se base sur la défaillance du système de gestion existant.
\newline Parmi ces defaillances: 
\begin{itemize}
  \item La majorité des ressources d'information sont sous forme de supports papier, d'où la difficulté dans le classement, l'archivage et le suivi;
  \item Difficulté du traitement à jours des documents, vue la difficulté de l’adaptation des support papiers;
  \item Difficultés de retrouver des données anciennes en un temps réduit;
  \item La charge du travail très importante vue l’absence d’un vrai système qui centralise les informations.
 \end{itemize} 
Au vue de toutes ses difficultés, l'application e-boutique s'inscrit comme une solution pour l'optimisation de la gestion de la boutique.

\subsection{Objectifs du projet}
Afin de remédier à tous ces problèmes, les objectifs visés par le projet sont :
\begin{itemize}
  \item Une gestion optimale des employés, leurs durées de travail ainsi que leurs congés;
  \item Une amélioration de la qualité des services;
  \item Un bon suivi des ventes, des dettes et du stock;
  \item Une optimisation des achats;
  \item Un gain en temps de reponse.
\end{itemize}
   
\subsection{Périmètre du projet}

\begin{itemize}
  \item L'application vise toutes les boutiques désirant numériser leur système de gestion.
  \item Le site sera disponible en français uniquement.
  \item L'ensemble des fonctionnalités seront accessibles depuis un mobile ou un desktop.
\end{itemize}


\newpage

\section{Spécification des besoins}

\subsection*{Introduction}
\addcontentsline{toc}{subsection}{Introduction}
Etant la première phase dans le cycle de developpement de toute application, cette phase consiste à comprendre le contexte du système.
\newline Nous allons ainsi dans un premier temps extraire les besoins fonctionnels et les besoins non fonctionnels. Nous allons ensuite determiné les acteurs les plus pertinents et identifier les cas
d'utilisation initiaux

\subsection{Les besoins fonctionnels}
Les différentes fonctionnalités de l'application décomposées sous forme de modules générales sont:
\begin{itemize}
  \item Gestion des ventes
  \item Gestion des dettes
  \item Gestion de congés
  \item Gestion de stock
  \item Gestion de personnel
\end{itemize}

\subsection{Les besoins non fonctionnels}
Les besoins non fonctionnels décrivent toutes les exigences internes afin de garantir un bon fonctionnement du système. Pour notre application, nous avons dégagé les besoins non fonctionnels suivants:
\newline
\newline \textbf{La sécurité :} Le système doit fournir le niveau de sécurité souhaité. Une fois la session détruite, l'auteur aura besoin de se ré-authentifier
\newline
\newline \textbf{Fiabilité :} L'application doit fonctionner de façon cohérente sans erreurs et satisfaisante

\subsection{Identification des acteurs}
Après les analyses faites au niveau du service, et après l’extraction et l’étude des besoins fonctionnels attendus par le système, on peut distinguer 2 types d’acteurs:
L'administrateur (responsable du service personnel) et le gérant (l'employé).

\subsection{Diagramme de cas d'utilisation général}
Le diagramme de cas d'utilisation représente la structure des grandes fonctionnalités du système.
\newline
\begin{figure}[ht]
\centering
\includegraphics[width=15cm, height=9cm]{UCG}
\caption{Diagramme de cas d'utilisation général}
\end{figure}
\newline
\newline Dans ce qui suit, nous expliquerons de façon detaillée chaque cas d'utilisation.

\subsubsection{Analyse du cas d'utilisation: gestion des ventes}
La figure ci-dessous représente le cas d'utilisation gestion des ventes.
\newpage
\begin{figure}[ht]
\includegraphics[width=15cm, height=9cm]{venteUC}
\caption{Diagramme de cas d'utilisation: gestion des ventes} 
\end{figure}
\begin{itemize}
  \item \textbf{Scénario du cas d’utilisation « gestion des ventes » :} 
\end{itemize}
Le tableau suivant résume les scénarii réalisés par les acteurs identifiés au niveau du cas
d’utilisation « gestion des ventes » ;
\newline
\newline
\begin{tabular}{|r|l|} \hline
\textbf{Acteurs} & Gérant\\ \hline
\textbf{Objectif} & \begin{minipage}{0.95\textwidth}
  \textbf{La gestion des ventes donne la main au gérant de :}
  \begin{itemize}
    \item Consulter la liste des ventes
    \item Modifier une vente
    \item Supprimer une vente
    \item Ajouter une vente
    \item Imprimer la facture
  \end{itemize}
\end{minipage} \\ \hline
\textbf{Précondition} & Authentification du gérant réussi et choix d'une opération \\ \hline
\textbf{Postcondition} & Opération finalisée et validée \\ \hline
\textbf{Scénario nominal} & \begin{minipage}{0.95\textwidth}
  \begin{enumerate}
    \item Le gerant s'authentifie au système et accède à son espace
    \item Il choisit l'option en rapport avec les ventes
    \item Il effectue ses opérations et quitte
\end{enumerate}
\end{minipage} \\ \hline
\end{tabular}


\subsubsection{Analyse du cas d'utilisation: gestion des dettes}
La figure ci-dessous représente le cas d'utilisation gestion des dettes.
\newpage
\begin{figure}[ht]
\includegraphics[width=15cm, height=9cm]{detteUC}
\caption{Diagramme de cas d'utilisation: gestion des dettes} 
\end{figure}
\begin{itemize}
  \item \textbf{Scénario du cas d’utilisation « gestion des dettes » :} 
\end{itemize}
Le tableau suivant résume les scénarii réalisés par les acteurs identifiés au niveau du cas
d’utilisation « gestion des dettes » ;
\newline
\newline
\begin{tabular}{|r|l|} \hline
\textbf{Acteurs} & Gérant\\ \hline
\textbf{Objectif} & \begin{minipage}{0.95\textwidth}
  \textbf{La gestion des ventes donne la main au gérant de :}
  \begin{itemize}
    \item Octroyer une dette
    \item Enrégitrer une dette reglée
    \item Consulter la liste des dettes
  \end{itemize}
\end{minipage} \\ \hline
\textbf{Précondition} & Authentification du gérant réussi et choix d'une opération \\ \hline
\textbf{Postcondition} & Opération finalisée et validée \\ \hline
\textbf{Scénario nominal} & \begin{minipage}{0.95\textwidth}
  \begin{enumerate}
    \item Le gerant s'authentifie au système et accède à son espace
    \item Il choisit l'option en rapport avec les dettes
    \item Il effectue ses opérations et quitte
\end{enumerate}
\end{minipage} \\ \hline
\end{tabular}

\subsubsection{Analyse du cas d'utilisation: gestion des congés}
La figure ci-dessous représente le cas d'utilisation gestion des congés.
\newpage
\begin{figure}[ht]
\includegraphics[width=15cm, height=9cm]{congeUC}
\caption{Diagramme de cas d'utilisation: gestion des congés} 
\end{figure}
\begin{itemize}
  \item \textbf{Scénario du cas d’utilisation « gestion des congés » :} 
\end{itemize}
Le tableau suivant résume les scénarii réalisés par les acteurs identifiés au niveau du cas d’utilisation « gestion des congés » ;
\newline
\newline
\begin{tabular}{|r|l|} \hline
\textbf{Acteurs} & Gérant et administrateur\\ \hline
\textbf{Objectif} & \begin{minipage}{0.95\textwidth}
  \textbf{La gestion des ventes donne la main à l'administrateur de :}
  \begin{itemize}
    \item Consulter la liste des congés
    \item Modifier/mettre à jour un congé
    \item Supprimer un congé
    \item Ajouter un nouveau congé
  \end{itemize}
  \textbf{Et au gerant de :}
  \begin{itemize}
    \item Consulter la liste des congés
    \item Suivre ses congés 
  \end{itemize}
\end{minipage} \\ \hline
\textbf{Précondition} & Authentification de l'utilisateur réussi et choix d'une opération \\ \hline
\textbf{Postcondition} & Opération finalisée et validée \\ \hline
\textbf{Scénario nominal} & \begin{minipage}{0.95\textwidth}
  \begin{enumerate}
    \item L'utilisateur s'authentifie au système et accède à son espace
    \item Il choisit l'option en rapport avec les congés
    \item Il effectue ses opérations et quitte
\end{enumerate}
\end{minipage} \\ \hline
\end{tabular}


\subsubsection{Analyse du cas d'utilisation: gestion des stock}
La figure ci-dessous représente le cas d'utilisation gestion des stock.
\newpage
\begin{figure}[ht]
\includegraphics[width=15cm, height=9cm]{stockUC}
\caption{Diagramme de cas d'utilisation: gestion des stock} 
\end{figure}
\begin{itemize}
  \item \textbf{Scénario du cas d’utilisation « gestion des stock » :} 
\end{itemize}
Le tableau suivant résume les scénarii réalisés par les acteurs identifiés au niveau du cas
d’utilisation « gestion des stock » ;
\newline
\newline
\begin{tabular}{|r|l|} \hline
\textbf{Acteurs} & Gérant et administrateur\\ \hline
\textbf{Objectif} & \begin{minipage}{0.95\textwidth}
  \textbf{La gestion des ventes donne la main à l'administrateur de :}
  \begin{itemize}
    \item Gerer les entrées et sorties du stock
    \item Planifier les livraisons
    \item Suivre l'état du stock
  \end{itemize}
  \textbf{Et au gerant de :}
  \begin{itemize}
    \item Gerer les entrées et sorties du stock
    \item Contrôler l'état du stock 
  \end{itemize}
\end{minipage} \\ \hline
\textbf{Précondition} & Authentification de l'utilisateur réussi et choix d'une opération \\ \hline
\textbf{Postcondition} & Opération finalisée et validée \\ \hline
\textbf{Scénario nominal} & \begin{minipage}{0.95\textwidth}
  \begin{enumerate}
    \item L'utilisateur s'authentifie au système et accède à son espace
    \item Il choisit l'option en rapport avec les congés
    \item Il effectue ses opérations et quitte
\end{enumerate}
\end{minipage} \\ \hline
\end{tabular}


\end{document}
